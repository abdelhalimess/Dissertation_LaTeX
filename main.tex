\documentclass[12pt]{report}
\usepackage{lmodern}
\usepackage{microtype}
\usepackage[utf8]{inputenc}
\usepackage[english]{babel}
\usepackage{hyperref}
\usepackage{graphicx}
\usepackage{float}
\usepackage{indentfirst}
\usepackage{url}
\usepackage{amsmath}
\usepackage[margin=2cm]{geometry}
\begin{document}
\begin{titlepage}
    \centering
    \textsc{\Large People's Democratic Republic of Algeria}\\[0.2cm]  % Algeria header
    \textsc{\Large Ministry of Higher Education and Scientific Research}\\[0.8cm]  % Ministry header
    
    \vspace{0.5cm}
    
    \includegraphics[width=3cm]{university-logo.png}
    
    \textbf{\Huge Hassiba Benbouali University}
    
    \vspace{0.5cm}
    
    \textbf{\LARGE Faculty of Exact Sciences And Informatics}
    \vspace{2cm}
    
    \rule{\linewidth}{1pt} % Horizontal line
    
    \vspace{0.5cm}
    
    \textbf{\LARGE Dissertation}
    
    \vspace{0.5cm}
    
    \textbf{\huge CNAS Virtual Counter}
    
    \vspace{0.5cm}
    
    \rule{\linewidth}{1pt} % Horizontal line
    
    \begin{minipage}{0.5\textwidth}
        \vspace{8cm}
        \begin{flushleft}
        \textbf{\Large By}\\
        \textbf{\Large Abdelhalim Esselami}\\
        \textbf{\Large Abdelkadir Cheklal}
        \end{flushleft}
        \end{minipage}%
        \begin{minipage}{0.5\textwidth}
        \vspace{8cm}
        \begin{flushright}
        \textbf{\Large Supervisors:}\\
        \textbf{\Large Mr. Walid Kadri}\\
        \textbf{\Large Mr. Abdellatif Esselami}
        \end{flushright}
        \end{minipage}
    
    \vspace{1cm}
    
    \rule{\linewidth}{1pt} % Horizontal line
    
    \vfill
    
    \textbf{\Large May 2023}
    
    \end{titlepage}

        \vspace*{\fill}
        \section*{\centering Acknowledgment}

        We would like to express our deepest gratitude to our supervisors, Mr. Walid Kadri and Mr. Abdellatif Esselami, for their guidance, support, and valuable insights throughout the duration of this dissertation. Their expertise and encouragement have been instrumental in the successful completion of this work.
        
        We are also grateful to our colleagues and friends for their assistance and collaboration during this research. Their contributions and discussions have greatly enriched the outcome of this study.
        
        Furthermore, we would like to acknowledge the support provided by the CNAS organization for their cooperation and provision of necessary resources for conducting this research.
        
        Finally, we extend our heartfelt appreciation to our families for their unwavering love, encouragement, and understanding throughout this academic journey.
        
        \vspace*{\fill}
        \thispagestyle{empty}
        \clearpage
    \begin{abstract}
        This dissertation presents a comprehensive study on the development and implementation of the CNAS Virtual Counter. The aim of this research is to provide an innovative solution for improving the efficiency and accessibility of counter services in the CNAS organization.
        
        The dissertation begins with an analysis of the existing counter system and identifies the limitations and challenges faced by both the organization and its clients. Subsequently, the design and architecture of the CNAS Virtual Counter are presented, highlighting the integration of advanced technologies such as web-based interfaces, real-time communication, and secure data management.
        
        A detailed description of the development process is provided, outlining the technologies, frameworks, and methodologies employed. The implementation of key features, including user registration, appointment scheduling, document submission, and online assistance, is thoroughly discussed.
        
        To evaluate the effectiveness of the CNAS Virtual Counter, a series of user studies and performance tests are conducted. The results demonstrate significant improvements in service efficiency, reduced waiting times, and enhanced user satisfaction compared to the traditional counter system.
        
        Furthermore, the dissertation addresses the security and privacy concerns associated with the virtual counter, presenting a comprehensive framework for data protection and access control. The implementation of encryption, user authentication, and secure storage measures ensures the confidentiality and integrity of sensitive information.
        
        Overall, the CNAS Virtual Counter proves to be a successful implementation that offers numerous benefits to both the organization and its clients. The research contributes to the advancement of counter service systems and sets a foundation for future enhancements and expansions.
        
        \end{abstract}
            \vspace*{\fill}

        \section*{\centering \large Resumé}
            Ce mémoire présente une étude approfondie sur le développement et la mise en œuvre du Guichet Virtuel CNAS. L'objectif de cette recherche est de fournir une solution innovante pour améliorer l'efficacité et l'accessibilité des services de guichet au sein de l'organisation CNAS.
            
            La mémoire commence par une analyse du système de guichet existant, identifiant les limitations et les défis auxquels sont confrontés l'organisation et ses clients. Ensuite, la conception et l'architecture du CNAS Virtual Counter sont présentées, mettant en évidence l'intégration de technologies avancées telles que les interfaces web, la communication en temps réel et la gestion sécurisée des données.
            
            Une description détaillée du processus de développement est fournie, en décrivant les technologies, les frameworks et les méthodologies utilisées. La mise en œuvre des fonctionnalités clés, notamment l'inscription des utilisateurs, la prise de rendez-vous, la soumission de documents et l'assistance en ligne, est discutée en détail.
            
            Pour évaluer l'efficacité du Guichet Virtuel CNAS, une série d'études utilisateurs et de tests de performance sont menés. Les résultats démontrent des améliorations significatives en termes d'efficacité du service, de réduction des temps d'attente et de satisfaction accrue des utilisateurs par rapport au système de guichet traditionnel.
            
            De plus, la mémoire aborde les problématiques de sécurité et de confidentialité associées au guichet virtuel, en présentant un cadre complet de protection des données et de contrôle d'accès. La mise en œuvre de mesures telles que le chiffrement, l'authentification des utilisateurs et le stockage sécurisé garantit la confidentialité et l'intégrité des informations sensibles.
            
            Dans l'ensemble, le Guichet Virtuel CNAS s'avère être une implémentation réussie offrant de nombreux avantages tant pour l'organisation que pour ses clients. Cette recherche contribue à l'avancement des systèmes de services de guichet et pose les bases pour de futures améliorations et extensions.
            \vspace*{\fill}
        \thispagestyle{empty}  
                    
\tableofcontents
\listoffigures
\listoftables


\chapter*{Abbreviations}

\begin{itemize}
    \item \textbf{CNAS} - National Social Security Fund (Caisse Nationale de Sécurité Sociale)
    \item \textbf{API} - Application Programming Interface
    \item \textbf{GUI} - Graphical User Interface
    \item \textbf{SQL} - Structured Query Language
    \item \textbf{HTML} - Hypertext Markup Language
    \item \textbf{CSS} - Cascading Style Sheets
    \item \textbf{JS} - JavaScript
    \item \textbf{PHP} - Hypertext Preprocessor
    \item \textbf{MVC} - Model-View-Controller
    \item \textbf{API} - Application Programming Interface
    \item \textbf{UI} - User Interface
    \item \textbf{UX} - User Experience
    \item \textbf{IDE} - Integrated Development Environment
    \item \textbf{VCS} - Version Control System
\end{itemize}


\addcontentsline{toc}{chapter}{General Introduction}
\section*{General Introduction}

The rapid advancements in information technology have transformed the way organizations operate and manage their processes. One such domain that has witnessed significant growth is the development of web applications for various purposes. In particular, web applications have become essential tools for improving the efficiency of service delivery and reducing operational costs for many organizations, including government institutions.

\addcontentsline{toc}{section}{Problematic and Objectives}
\section*{Problematic and Objectives}
The National Social Security Fund ( CNAS : Caisse Nationale des Assurances Sociales des Travailleurs Salariés) in Algeria is one such organization that can benefit from the adoption of web-based solutions as it is responsible for providing a range of social security services to Algerian citizens, including health insurance, retirement benefits, and unemployment benefits. The organization serves a large number of people and has several applications that we will talk about in upcoming chapters. 

Although the current system of managing queues at CNAS helps with the organization and the process of the work, it has proved to be inefficient and time-consuming for both the employees and the beneficiaries of social security, and it has been struggling to keep up with the increasing demand. For instance, imagine coming all the way to CNAS and having to wait for an hour just to get information about a document, knowing that it could be obtained in seconds through a web-based solution. This highlights the inefficiency of the current system, which is not only time-consuming but also inconvenient for the beneficiaries who have to take time off from work to visit CNAS. A web-based solution that streamlines the appointment management process will save time and effort for both the employees and the beneficiaries and will enhance the overall efficiency of the services provided by CNAS.

Therefore, the objective of this project is to create a web application that streamlines the appointment management process to improve the overall efficiency of the services provided by CNAS. The proposed web application has a key feature that enables users to choose the service and the task they want to do at CNAS before booking an appointment. At the beginning of the user's journey through the application, they are prompted to complete a questionnaire that helps generate a personalized Checklist of the necessary documents and steps they need to complete in order to achieve their goal. This questionnaire feature streamlines the process for the user by providing clear guidance and ensuring that no important documents or steps are missed. This feature ensures that the user is directed to the appropriate service desk for their needs, reducing the time wasted on unnecessary visits and allowing users to access all the necessary information online and plan their appointments accordingly.

Moreover, this web application will include a range of features designed to enhance the appointment management process, including the ability to track the status of appointments and documents, a reminder and notification system, customization of appointments and schedules, an authentication and a solid security system.

%In addition, the COVID-19 pandemic has highlighted the need for remote access to services to reduce physical interactions and ensure social distancing.
\newpage
\addcontentsline{toc}{section}{Dissertation Plan}

\section*{Dissertation Plan}

This dissertation is structured into three main chapters. Chapter one provides an in-depth analysis of the state of the art in virtual counters and appointment management systems, highlighting the existing solutions, challenges, and limitations. Chapter two focuses on the conception phase, where the requirements for the virtual counter system are identified, user needs are analyzed, and the system design and architecture are described. Chapter three delves into the implementation phase, discussing the practical aspects of developing the virtual counter system, including GitHub setup and configuration, development methodologies, and integration of relevant technologies. The dissertation concludes with a comprehensive evaluation of the implemented system, future recommendations for enhancements, and the significance of the findings. Throughout the dissertation, a thorough examination of the relevant literature and appropriate methodologies is conducted to ensure the research objectives are met and to provide valuable insights into the field of virtual counters and appointment management systems.


\chapter{State of The Art}
\section{Introduction to CNAS Organization}

The Caisse Nationale des Assurances Sociales (CNAS) is a public institution in Algeria responsible for managing the social security system. Established in 1967, CNAS provides social security benefits and services to workers, retirees, and their dependents in Algeria. The organization operates under the Ministry of Labor, Employment and Social Security and is governed by a board of directors appointed by the Algerian government.

CNAS offers a range of social security services to its beneficiaries, including health insurance, maternity benefits, disability benefits, and pensions. The organization's services are essential for the Algerian society, as they provide financial protection and support to workers and their families.



\chapter{Conception}
\section{Purpose of the chapter}
The purpose of this chapter is to present the conception of a virtual counter system for the Algerian National Social Security Fund (CNAS). This chapter will provide a detailed explanation of the system design and architecture, database design, as well as the different diagrams and models used during the conception phase. The virtual counter system aims to improve the current management system used by CNAS by providing users with a more efficient and user-friendly way to gather necessary information and book appointments.
\section {Overview of the topics covered}
This chapter focuses on the conception of the virtual counter system for CNAS. It includes the analysis and design of the system, from the identification of user requirements to the development of the system architecture and database design. The chapter also includes the presentation of the different diagrams that were created, such as the use case diagram, class diagram, sequence diagram, and flowchart.
The aim of this chapter is to provide a comprehensive understanding of the virtual counter system, its components, and its functionalities.

\section{System design and architecture}
The system design and architecture of a virtual counter is a crucial aspect in developing a successful web application. It involves designing the components of the system and specifying how they interact with each other to achieve the desired functionality. In the case of a virtual counter for CNAS, the system design and architecture must take into account the different types of users, such as clients and agents, and the various tasks they need to perform. It must also ensure that the application is secure and reliable, with measures in place to protect user data and prevent unauthorized access. The system design and architecture will involve selecting suitable technologies and frameworks, such as Laravel and VueJs, and designing a database schema to store and retrieve data efficiently. Overall, a well-designed system architecture will contribute to the effectiveness and efficiency of the virtual counter and improve the user experience for both clients and agents.

\subsection{Description of the overall system architecture}
The overall system architecture of the virtual counter for CNAS is designed to be a web-based application with a client-server architecture. The client-side will be a user-friendly interface, developed using Vue.js framework, that allows users to interact with the system and perform different tasks, such as filling in a questionnaire that will generate a checklist of required documents, booking appointments, and checking their status. On the other hand, the server-side of the application will handle all the processing and data storage. It will be developed using the Laravel framework, which is a powerful and reliable PHP web application framework that enables rapid application development with a robust and scalable codebase. The application will also use a MySQL database to store all the necessary data, such as user information, appointment schedules, and queue status. The overall system architecture is designed to be modular and scalable, allowing for easy maintenance and future updates.
\section{Diagrams illustrating the different components of the system}
Diagrams can help to provide a visual representation of the different components and processes involved in the virtual counter system, making it easier to understand and communicate to stakeholders.

\medskip The use of UML (Unified Modeling Language) which is a standered Language for visualizing and creating views to illustrate the different parts of a system , presenting us with a various types of diagrams that facilitates the conception phase for the virtual counter and makes it more comprehensive .  

\subsection{Use case diagram}
Use case diagram is one of the most used static diagrams in UML , it consist on explaining the different actions preformed by the user and helps understanding the main functions that can be preformed by the system.  

When the user is interacting with the system, the virtual counter enables him to consult the various services provided  by CNAS without the need to log in. 
 
 Additionally, the user can also complete a variety of tasks, such as selecting a service and completing a questionnaire related to that service. The system will then generate a checklist of the documents he will need to submit. The user can stop at printing that checklist or he can move on to booking an appointment which will require him to be authenticated. When an appointment is booked, an appointment ticket, that contains the previous checklist along with some appointment details such as the date and time, the counter number and the name of the employee responsible for treating your concerns,  will be available to print. 
 
 \medskip In the second hand of the virtual counter, both the employee and the supervisor have their own interactions with the system; however, in both their cases, they both need to be logged in order to access the various functionalities of the system. In addition to managing their work flow, both can manage the appointments by treating them and updating the queue. 
\newpage
 \medskip Here is the diagram:

 \begin{figure}[H]
    \centering
    \includegraphics[width=0.90\textwidth]{useCASE.png}
    \caption{Use case diagram}
    \label{ucdiagram}
 \end{figure}
 
\subsection{Class diagram}
Class diagram is one of the static diagrams used to illustrate the overall architecture of the classes utilized in the application , it helps developers to capture the global planing of the project in which  will simplify the development in the first phases and the maintenance in the final product . 

\medskip On the same token , a collection of a class diagram represents the whole system . In addition the virtual counter conception we focused on tracing the whole activities of the user and the CNAS employee , so that it will be easy and much more efficient for the supervisor to track and see the statistics of the application . 

\begin{figure}[H]
    \centering
    \includegraphics[width=1.0\textwidth]{class_diagram.PNG}
    \caption{Class diagram}
    \label{fig:Class diagram }
\end{figure}

\subsection{Sequence diagram}
Sequence diagram in one of the well known dynamic diagrams that allow the overall understanding for the hidden functionalities , and streamlines the developemnt phase . 

\medskip Here is the different diagrams related to every user in the application as well as the registration sytem . 
  \begin{figure}[H]
      \centering
      \includegraphics*[width=0.6\textwidth]{registration_sequence.PNG}
      \caption{Sequence diagram for Registration}
      \label{fig:Sequence diagram for Registration}
  \end{figure}
  \begin{figure}[H]
      \centering
      \includegraphics*[width=0.6\textwidth]{sequence_client.PNG}
      \caption{Sequence diagram for client interaction}
      \label{fig:Sequence diagram for Cient Interaction}
  \end{figure}
 
\subsection{Discussion of the design decisions made}
On the whole idea of  eliminating the wating time in such an efficient maner and to guarantee a powerful system , we reached to a finale conception of the virtual counter, which can assure by far a perfect user experience and a robust system in both sides as beneficiary or an employee . 

\medskip The use of this perticular system can be beneficial since we focused on creating a web application that covers all the problems that may be incontered along the working time of the application . 
 
For instence , if we consider that the patient want to get information it not recommended to force him to create an account since it's an information gathering , although booking an appointment in one of the services require the use of the patient information such as the name , adress ... ect , so it recommended to create an account that assures the integrity of his identity .    
\section{Database design}
In this juncture of the conception our main focus was on designing a reliable data base that can satisfy the needs of the virtual counter , along with offering the stakeholders the possibility to track and store the data efficiently and making sure that accessing the data will be in a secure process . 

//////
/////
\subsection{Overview of the database schema}
//
\subsection{Explanation of the different tables and their relationships}
//
\subsection{Discussion of the design decisions made}

//
\chapter{Implementation}
\section{Chapter Overview}
In this chapter, we will delve into the comprehensive implementation details of CNAS's virtual counter system. We will explore the utilization of various cutting-edge technologies that have played a pivotal role in the development process. Firstly, we will highlight the significance of Version Control Tool GitHub, which facilitated seamless collaboration and ensured efficient code management throughout the project. Additionally, we will examine the utilization of the widely acclaimed PHP framework Laravel, known for its robustness and flexibility, which provided a solid foundation for building the web application. Furthermore, we will explore the integration of VueJs, a powerful JavaScript framework, that enabled us to develop an interactive and user-friendly interface. Together, these technologies synergistically contributed to the creation of a highly efficient and functional virtual counter system for CNAS.
\section{Introduction to Git }
Software development involves managing a large number of files and assets that undergo frequent changes. As developers, we require a tool that facilitates the administration of these files and ensures consistent updates. This is where Git proves invaluable, providing us with the ability to handle such tasks with ease and flexibility. At its core, Git is a powerful tool that enables multiple individuals to collaborate on the same project while effectively tracking all changes made to the code and files over time.
\subsection{GitHub implementation}
In order to facilitate the development phase and ensure efficient version control for our virtual counter project, we have implemented GitHub. GitHub is a web-based platform that serves as a central repository for Git-based version control systems. It provides a range of tools and features that enable collaborative development, code management, and tracking of changes.

With GitHub, we have a centralized location where we can store and manage our project's codebase. It allows us to create and manage repositories, branches, and commits, making it easy to track changes and work on different features or bug fixes simultaneously. GitHub's version control capabilities ensure that we have a complete history of all modifications, allowing us to roll back changes if needed and maintain code integrity.

GitHub also offers collaborative features that enhance team collaboration and communication. We can create issues and assign them to team members, facilitating task management and bug tracking. Additionally, GitHub provides a platform for code review, allowing team members to review and provide feedback on each other's code, ensuring code quality and consistency.

By utilizing GitHub, we benefit from a robust and scalable infrastructure for our project's version control needs. It streamlines our development process, enables efficient collaboration, and ensures the traceability and integrity of our codebase.
\subsection{Advantages of Github in Development}
In this section, we will explore the advantages of utilizing GitHub in our development workflow. GitHub, as a powerful version control system and collaboration platform, offers a range of benefits that enhance the efficiency and effectiveness of our project.

\begin{itemize}
\item \textbf{Version Control:} GitHub allows for efficient and effective version control, enabling easy tracking of changes, branching, and merging of code. This ensures that the project's codebase is well-managed and allows for easy collaboration among team members.

\item \textbf{Collaboration and Teamwork:} GitHub provides a platform for seamless collaboration and teamwork. It allows multiple developers to work on the same project simultaneously, facilitating efficient communication, code sharing, and coordination of tasks. Features like pull requests and code reviews enhance collaboration and ensure high code quality.

\item \textbf{Code Integrity and History:} GitHub maintains a complete history of all code changes, making it easy to track modifications, roll back to previous versions if necessary, and maintain code integrity. This helps in identifying and resolving issues, ensuring a stable and reliable codebase.

\item \textbf{Project Management:} GitHub offers project management features such as issue tracking, task assignment, and milestone tracking. These tools streamline project management, enhance organization, and ensure that tasks are tracked and completed in a timely manner.

\item \textbf{Community and Open Source Collaboration:} GitHub has a large community of developers and provides a platform for open-source collaboration. It enables easy sharing of code, contribution to open-source projects, and learning from others in the community.
\end{itemize}
\subsection{GitHub Setup and Configuration}

In order to effectively utilize the features of GitHub for version control and collaboration, it is necessary to set up a GitHub account and configure Git on your local machine. This section provides step-by-step instructions on how to set up and configure GitHub, enabling you to seamlessly manage and contribute to your project repositories. Follow the steps below to get started:

\begin{enumerate}
    \item \textbf{Create a GitHub Account}: Begin by creating a GitHub account. Visit the GitHub website (\url{https://github.com}) and sign up for a new account. Provide the required information, such as your username, email address, and a secure password. Once registered, verify your email address to activate your GitHub account.
    
    \item \textbf{Install Git}: Proceed to install Git on your local machine if you haven't done so already. Git provides the necessary command-line tools to interact with GitHub repositories. Download the Git installer from the official website (\url{https://git-scm.com/downloads}) and follow the installation instructions for your operating system.
    
    \item \textbf{Configure Git}: After installing Git, configure your Git identity by setting your username and email address. Open the command-line interface (e.g., Terminal, Git Bash) and execute the following commands:
    
    \begin{verbatim}
    $ git config --global user.name "Your Name"
    $ git config --global user.email "your-email@example.com"
    \end{verbatim}
    
    These settings will be associated with your Git commits and will be visible in the commit history.
    
    \item \textbf{Generate SSH Key}: For secure interaction with GitHub repositories, it is recommended to generate an SSH key pair. Generate a new SSH key by executing the following command:
    
    \begin{verbatim}
    $ ssh-keygen -t rsa -b 4096 -C "your-email@example.com"
    \end{verbatim}
    
    Follow the prompts to specify the location for storing the key pair and provide a passphrase (optional but recommended). Once generated, add the SSH public key to your GitHub account by navigating to "Settings" -> "SSH and GPG keys" and adding the public key.
    
    \item \textbf{Configure Remote Repository}: If you are collaborating on an existing GitHub repository, clone the repository to your local machine using the following command:
    
    \begin{verbatim}
    $ git clone git@github.com:username/repository.git
    \end{verbatim}
    
    Replace \texttt{username} with your GitHub username and \texttt{repository} with the name of the repository. This command creates a local copy of the repository on your machine.
\end{enumerate}

By following these steps, you will have successfully set up and configured GitHub for your project, empowering you to effectively manage version control and collaborate with others in your development process.

\subsection{Github commands}
GitHub provides a powerful set of commands that enable efficient collaboration and version control in software development projects. These commands allow developers to clone repositories, create and manage branches, commit changes, push and pull code, merge branches, and initiate pull requests. Understanding these essential commands is crucial for effective GitHub usage and seamless teamwork. In this section, we will explore the key GitHub commands along with their descriptions and usage examples.

\medskip \begin{itemize}
    \item \textbf{Clone:} The \texttt{git clone} command allows you to create a local copy of a remote repository on your computer. For example: \texttt{git clone <repository URL>}.
  
    \item \textbf{Branch:} The \texttt{git branch} command is used to create, list, or delete branches in your repository. For example: \texttt{git branch <branch name>}.
  
    \item \textbf{Commit:} The \texttt{git commit} command is used to save changes made to your local repository. For example: \texttt{git commit -m "Commit message"}.
  
    \item \textbf{Push:} The \texttt{git push} command is used to upload local repository commits to a remote repository. For example: \texttt{git push origin <branch name>}.
  
    \item \textbf{Pull:} The \texttt{git pull} command is used to update your local repository with the latest changes from the remote repository. For example: \texttt{git pull origin <branch name>}.
  
    \item \textbf{Merge:} The \texttt{git merge} command is used to combine changes from different branches into the current branch. For example: \texttt{git merge <branch name>}.
  
    \item \textbf{Pull Request:} The \texttt{git pull request} command is used to propose changes from a branch to be merged into another branch. For example: \texttt{git pull request}.
  \end{itemize}
  \medskip \textbf{SCREENSHOTS : GIT BASH, VS CODE TERMINAL }

  \subsection{Collaboration and Teamwork}
GitHub provides powerful features that enable seamless collaboration and effective teamwork on software development projects. This section explores the various collaborative capabilities offered by GitHub, allowing multiple developers to work together efficiently and coordinate their efforts. From managing branches and pull requests to resolving conflicts and conducting code reviews, GitHub facilitates a collaborative environment that fosters teamwork and enhances productivity. This section demonstrates how to leverage these collaborative features to streamline the development process and maximize the effectiveness of your team.   

\medskip \textbf{SCREENSHOTS : ISSUES, Pull requests and code reviews, project boards, Commenting..}
\subsection{Conclusion}
In this section, we explored the implementation of GitHub as a powerful collaboration and version control tool for our project. We discussed the setup and configuration process, essential commands for managing repositories, and the benefits of using GitHub for collaboration and teamwork. By leveraging GitHub's features such as branch management, pull requests, issue tracking, and project boards, we have enhanced our team's productivity and streamlined our development process. The use of GitHub has enabled us to effectively collaborate, track changes, and ensure the integrity of our codebase. With its robust features and user-friendly interface, GitHub has become an indispensable tool for our project's success.
\section{Introduction to Laravel framework}
Laravel is one of the most well known web frameworks that is used widely among developers, it is an open-source PHP based framework that uses MVC (Modal-View-Controller) Architecture and offers various tool and features that allows developers to build high-quality applications with such an efficiency and quickness. 

\medskip Benefits of using Laravel for web development:
\begin{itemize}

\item \textbf{Expressive syntax: }Laravel offers an expressive and readable syntax that simplifies the process of writing code. It provides a wide range of functions and shortcuts that allow developers to accomplish complex tasks with minimal effort.

\item \textbf{MVC architecture: }Laravel follows the Model-View-Controller (MVC) architectural pattern, which promotes separation of concerns and enhances code organization. This architectural approach enables developers to create modular and maintainable applications.

\item \textbf{Powerful ORM: }Laravel's Eloquent ORM (Object-Relational Mapping) simplifies database operations by providing an intuitive and fluent interface to interact with databases. It allows developers to work with database records as objects, making database management and querying a breeze.

\item \textbf{Robust routing system: }Laravel's routing system allows developers to define clean and flexible routes for their web applications. It supports various HTTP methods, route parameters, and route grouping, making it easy to handle complex URL structures.

\item \textbf{Blade templating engine: }Laravel's Blade templating engine offers a concise and powerful way to create dynamic views. It provides features like template inheritance, control structures, and reusable components, enabling developers to build modular and reusable UI components.

\item \textbf{Authentication and authorization: }Laravel simplifies user authentication and authorization processes with built-in functionalities. It provides secure user registration, login, and password reset mechanisms, as well as fine-grained access control using gates and policies.

\item \textbf{Rich ecosystem and community support: }Laravel has a vibrant and active community of developers who contribute to its growth. The framework benefits from a vast ecosystem of packages and libraries that extend its capabilities, allowing developers to leverage existing solutions and accelerate development.

\item \textbf{Testing and debugging tools: }Laravel provides robust testing and debugging tools that help developers ensure the quality and reliability of their applications. It supports unit testing, feature testing, and includes convenient debugging tools for efficient troubleshooting.
\end{itemize}

\medskip Overall, Laravel is an excellent choice for building web applications of any size and complexity, the choice of implementing this particular framework has been proven to be a wise decision, and that's due to its powerful set of tools an features that enabled us to create a robust and scalable web application that meets the needs of CNAS and its users. 
\section{Laravel implementation}
\subsection{Installation and setup}
In this section we will discuss the installation guide for laravel and its different components. In order to install and setup laravel correctly and without any issues, there is some requirements needs to be fulfilled in case of not using the homestead virtual machine. 

\medskip These requirements are 
\begin{itemize}
    \item PHP version 7.2.5 or greater. 
    \item BCMath PHP Extension
    \item Ctype PHP Extension
    \item Fileinfo PHP extension
    \item JSON PHP Extension
    \item Mbstring PHP Extension
    \item OpenSSL PHP Extension
    \item PDO PHP Extension
    \item Tokenizer PHP Extension
    \item XML PHP Extension
\end{itemize}
 \medskip If the previous requirements are validated, we then move on to the installation guide for laravel, to do so it is highly recommended to follow the steps listed.
 \begin{enumerate}
\item \textbf{Install Laravel and its dependencies:}
  
Ensure that you have PHP installed on your system. Laravel requires PHP 7.4 or higher.
  
Install Composer, a dependency manager for PHP, if you haven't already. Composer is used to install Laravel and manage its dependencies.
  Open a terminal or command prompt and run the following command to install Laravel globally on your system:
   \begin{verbatim}
    composer global require laravel/installer
   \end{verbatim}
   

\item \textbf{Configure the development environment:}
  
Laravel requires a web server and a database to run. You can use popular web servers like Apache or Nginx, along with databases like MySQL or SQLite.
  
Ensure that your web server and database server are properly installed and configured. If needed, consult their respective documentation for installation and setup instructions.
\newpage
\item \textbf{Initialize a new Laravel project:}
 
Once Laravel is installed and your development environment is set up, you can create a new Laravel project.
  
Open a terminal or command prompt and navigate to the directory where you want to create your project.
  Run the following command to create a new Laravel project:
   \begin{verbatim}
    laravel new your-project-name 
   \end{verbatim}
   
   Replace "your-project-name" with the desired name for your project. This command will create a new directory with the specified project name and install the necessary files and dependencies.

\item \textbf{Test your installation:}
  
Change into the project directory:
\begin{verbatim}
    cd your-project-name
\end{verbatim}
   
Start the local development server by running the following command:
\begin{verbatim}
    php artisan serve
\end{verbatim}
  By default, the development server will start on (\url{http://localhost:8000}) 
  
  Open your web browser and visit that URL. If you see the Laravel welcome page, it means your installation was successful. 
  \item \textbf{PHP configuration:}
  
  Open the PHP configuration file (php.ini) on your system. The location of this file may vary depending on your operating system and PHP installation.
 
  Ensure that the following PHP extensions are enabled by uncommenting their respective lines (remove the semicolon ";" at the beginning of the line if present):
 \begin{verbatim}
    extension=fileinfo
    extension=openssl
    extension=pdo_mysql
 \end{verbatim}
 \newpage
 \item \textbf{Creating the ".env " file :}

 Laravel comes with a `.env.example' file by default. 
 \begin{itemize}
    \item 
    Make a copy of this file and rename it to `.env' by running the following command:
    \begin{verbatim}
       cp .env.example .env 
    \end{verbatim}
    \item Setting up the environment variables:
     
    Open the `.env' file in a text editor.

    Update the variables according to your development environment. For example, you might need to set the database credentials:
\begin{verbatim}
    DB_CONNECTION=mysql
    DB_HOST=127.0.0.1
    DB_PORT=3306
    DB_DATABASE=your_database_name
    DB_USERNAME=your_database_username
    DB_PASSWORD=your_database_password
\end{verbatim}

You can also configure other variables like the application URL, mail settings, caching drivers, and more. Refer to the comments in the `.env.example' file or Laravel's documentation for more information on available options.
\end{itemize}
\item \textbf{Generating the application key: }

Laravel requires an application key for secure encryption and other purposes. 

Run the following command to generate the key:
\begin{verbatim}
php artisan key:generate
\end{verbatim}
\item \textbf{Protecting sensitive information:}

   Ensure that the .env file is not publicly accessible. It should be kept outside of your version control system or any public directories.
    
   If you deploy your application to a production server, you may need to set the environment variables directly on the server or through the server's configuration management tools.
 \end{enumerate}
\subsection{Laravel directory structure}

In a Laravel project, the directory structure is designed to provide a clear organization for your application's files. Understanding the key directories and files will help you navigate and manage your Laravel project effectively. Here's an explanation of the purpose of each directory and some important files:
\begin{itemize}
    \item \textbf{app:}
        Contains the core application code, including controllers, models, and other PHP classes specific to your application's domain logic.

    \item \textbf{bootstrap:}
        Contains the files responsible for bootstrapping the Laravel framework and initializing the application environment.

    \item \textbf{config:}
        Contains configuration files for various aspects of your application, such as database connections, caching, mail settings, and more.

    \item \textbf{database:}
        Contains database-related files, including migrations for managing database schema changes, seeders for populating the database with sample data, and factories for generating test data.

    \item \textbf{public:}
        The web server's document root should be set to this directory. It contains the entry point for your application (index.php) and serves as the public-facing directory for static assets, such as CSS, JavaScript, and image files.

    \item \textbf{resources:}
        Contains views, language files, and frontend assets used by your application.
        views: Contains the Blade templates that define the UI of your application.
        lang: Contains language files for localization and internationalization.
        assets: Contains frontend assets, such as CSS, JavaScript, and images, that will be compiled and optimized by Laravel Mix.

    \item \textbf{routes:}
        Contains route definition files that specify how incoming requests should be handled by your application.
        web.php: Defines routes for web-based endpoints.
        api.php: Defines routes for API endpoints.
        You can create additional route files for organizing routes based on specific functionalities or modules.

    \item \textbf{storage:}
        Contains files generated by your application, such as logs, cached views, and uploaded files.
        app: Contains files generated by your application, such as cached config files, logs, and other temporary files.
        framework: Contains framework-generated files, including cached views, sessions, and routes.
        logs: Contains log files generated by your application.

    \item \textbf{tests:}
        Contains test files and directories for automated testing of your application.
        Feature: Contains feature tests, which test the application's behavior from the user's perspective.
        Unit: Contains unit tests, which test individual components of your application in isolation.

    \item \textbf{.env:}
        The environment file that holds environment-specific configuration values for your application.
        Contains settings such as database connections, mail configurations, and environment variables.
        It's important to keep this file secure and not expose any sensitive information.

    \item \textbf{composer.json and composer.lock:}
        These files manage the project's dependencies using Composer, a PHP dependency manager.
        composer.json lists the project's dependencies and defines autoloading rules.
        composer.lock locks the versions of the dependencies to ensure consistent installations.

    \item \textbf{artisan:}
        The command-line interface (CLI) tool for executing various commands within your Laravel application.
        Allows you to run tasks such as running migrations, generating code, and running tests.
\end{itemize}
Understanding the purpose of each directory and file in a Laravel project will help you navigate and locate the appropriate locations for adding or modifying code, configurations, and assets. It's important to maintain the integrity of the directory structure while organizing your code and assets within the appropriate directories.
\subsection{Routing}

Routing is an essential aspect of web development, and Laravel provides a powerful and flexible routing system. Here's an explanation of routing in Laravel, including how to define routes, work with route parameters, and utilize route grouping and naming.

Routing in Laravel refers to the process of mapping incoming HTTP requests to specific actions or handlers within your application. It determines how different URLs are handled and defines the endpoints through which users can access various functionalities of your application.

In Laravel, routes are typically defined in the `routes` directory, specifically the `web.php` and `api.php` files.
\begin{enumerate}
    \item \textbf{Basic route definition:} 
    
    A basic route is defined using the `Route` facade's methods, such as `get`, `post`, `put`, `patch`, and `delete`.
    
    \medskip Here's an example of a basic route definition:
    \begin{verbatim}
        Route::get('/home', function () {
            return 'Welcome to the home page!';
            });       
         \end{verbatim}
     
   This route responds to the `GET` request to the `/home` URL and returns the specified message.

 \item \textbf{Route parameters:}
   You can define routes with parameters that are passed as segments in the URL.
   
   \medskip Here's an example of a route with a parameter:
     \begin{verbatim}
        
        Route::get('/users/{id}', function ($id) {
            return 'User ID: ' . $id;
            });
        \end{verbatim}

    This route matches URLs like `/users/1`, `/users/2`, etc., and the parameter `{id}` is passed to the route closure as an argument.

\item \textbf{Route grouping and naming:}

Laravel allows you to group related routes and assign names to them for easy referencing and organization.
\begin{itemize}
    \item \textbf{Route grouping:}
    
    Route grouping allows you to apply common attributes or middleware to a group of routes.
    
    \medskip Here's an example of route grouping with a shared middleware:
\begin{verbatim}
    
        Route::middleware('auth')->group(function () {
            Route::get('/dashboard', function () {
                return 'Welcome to the dashboard!';
                });
                Route::get('/profile', function () {
                    return 'Welcome to your profile!';
                    });
                    });
                    
    \end{verbatim}
   In this example, the routes `/dashboard` and `/profile` are grouped together and share the `auth` middleware, which ensures that only authenticated users can access them.

    \item \textbf{Route naming:}
   
    Assigning names to routes helps in referencing them within your application, such as generating URLs or redirecting to specific routes.
    
    \medskip Here's an example of naming routes:
    \begin{verbatim}
        
        Route::get('/posts', function () {
            return 'List of posts';
            })->name('posts.index');
            
            Route::get('/posts/{id}', function ($id) {
                return 'Post ID: ' . $id;
                })->name('posts.show');
            \end{verbatim}
                
            In this example, the routes `/posts` and `/posts/{id}` are named as `posts.index` and `posts.show`, respectively. These names can be used later to generate URLs or redirect to these routes.
            
            By understanding and utilizing routing in Laravel, you can define the endpoints for your application, handle various HTTP methods, work with dynamic route parameters, group related routes, and assign names for easy referencing. Laravel's routing system provides the flexibility and convenience required to build robust and maintainable web applications.
        \end{itemize}
 \end{enumerate}
\subsection{Controllers}

Controllers play a crucial role in Laravel applications as they handle the logic and actions associated with different routes. Here's an explanation of creating and using controllers in Laravel, defining controller methods and actions, and understanding the separation of concerns between routes and controllers.

The creation of an controller is done by the following artisan command: 
\begin{verbatim}
   php artisan make:controller
\end{verbatim}

For instance, to create the user controller we used the following command: 
\begin{verbatim}
    php artisan make:controller UserController 
\end{verbatim}

This command will generate a new `UserController` class in the `app/Http/Controllers` directory. 
\subsection{Views and Blade templates}
//
        
\subsection{Models and Eloquent ORM}
//
\subsection{Database Migrations and Seeders}
//
\subsection{Form Handling and Validation}
//
\subsection{Authentication and Authorization}
//
\subsection{Middlewares}
//
\subsection{Error Handling and Logging}
//
\subsection{Testing in Laravel}
//
\subsection{Deployment and Production Considerations}
//
\subsection{Best Practices and Tips}
//
\subsection{Conclusion}
//
\section{VueJs implementation}
//
\subsection{Overview of the VueJs framework}
//
\subsection{Explanation of the different components of the system implemented using VueJs}
//
\subsection{Code snippets and screenshots to illustrate the implementation details}
//
\subsection{Discussion of the challenges faced and how they were overcome}
//


\section{Integration of Laravel and VueJs}
//
\subsection{Explanation of how Laravel and VueJs were integrated to create the final system}
//
\subsection{Code snippets and screenshots to illustrate the integration details}
//
\subsection{Discussion of the challenges faced and how they were overcome}

//
    \section{Conclusion}
//
    \subsection{Summary of the key points covered}
//
    \subsection{Reflection on the overall implementation process}
//
    \subsection{Discussion of future work and potential improvements}

    //
\addcontentsline{toc}{chapter}{Conclusion}
\chapter*{Conclusion}
After conducting a detailed analysis of the State of the Art, Conception, and Implementation of the CNAS Virtual Counter, it is evident that this innovative solution has the potential to revolutionize the way organizations manage their waiting queues.

The CNAS Virtual Counter is an effective web application built using Laravel and Vue.js that allows users to book appointments and gather information remotely. This application significantly reduces waiting time, providing users with a more convenient and efficient experience.

Through the use of innovative technologies, the CNAS Virtual Counter is an example of how organizations can leverage digital solutions to enhance customer experience and improve operational efficiency. By providing a platform that eliminates waiting time and enhances customer satisfaction, organizations can build a competitive advantage and improve their bottom line.

One of the key advantages of our application is its scalability and adaptability. This web application can be easily upgraded and customized to meet the needs of stakeholders and users. By incorporating feedback and suggestions from stakeholders and users, the CNAS Virtual Counter can continue to evolve and improve over time. This approach ensures that the application remains relevant and effective in addressing the needs of the organization and its customers.

In addition to its current capabilities, the CNAS Virtual Counter has the potential for further improvements and enhancements to better serve its users and meet the evolving needs of CNAS. One area of improvement is the implementation of multilingual support, allowing users to interact with the application in their preferred language. This would enhance accessibility and inclusivity, catering to a diverse user base.

Another valuable enhancement would be the introduction of schedule customization features. This would enable users to select specific time slots or preferred service providers based on their availability, optimizing the appointment booking process and further reducing waiting time.

Furthermore, integrating the CNAS Virtual Counter with CNAS's existing queue system would provide seamless coordination between the virtual and physical counter operations. This integration would allow real-time updates on queue status, enabling users to make informed decisions and better plan their visits to CNAS centers.

These future improvements would contribute to the continuous evolution and effectiveness of the CNAS Virtual Counter, aligning it with the dynamic needs of CNAS and its users. By embracing multilingual support, schedule customization, and integration with the existing queue system, CNAS can further enhance customer experience, improve operational efficiency, and reinforce its position as a leader in innovative service delivery.

In conclusion, the CNAS Virtual Counter is a valuable innovation that plays a significant role in eliminating waiting time. Its effectiveness is demonstrated by its successful implementation and its potential to be replicated by other organizations. The success of this application reinforces the importance of digital solutions in enhancing customer experience and operational efficiency.


\bibliography{references}
\bibliographystyle{ieeetr} 
\end{document}