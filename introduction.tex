\section{Introduction}
The present era has seen rapid advancements in information technology, transforming the way organizations operate and manage their processes. One such domain that has witnessed significant growth is the development of web applications for various purposes. This project focuses on creating a web application for the management of appointments at the National Social Security Fund (Caisse nationale des assurances sociales) in Algeria.

The main objective of this project is to improve the efficiency of the services provided by the CNAS through better appointment management. The current system of ticketing has proved to be inefficient and time-consuming for both the employees and the users of social security services. The choice of this subject stems from a desire to provide a more user-friendly and efficient solution to manage appointments and reduce the time wasted on simple tasks, such as obtaining information about the necessary documents for insurance applications.

One of the main challenges faced by the CNAS in appointment management is the time wasted by users visiting the CNAS offices to obtain information about required documents for insurance applications. A web application can address this issue by providing a dedicated platform that offers features like a To-Do List, allowing users to access all the necessary information online and plan their appointments accordingly.

The web application aims to offer various features to streamline the appointment management process, including:
\begin{itemize}
    \item Tracking the status of appointments and documents
    \item A reminder and notification system
    \item Parametrability of appointments and day shifts
    \item Appointment search and filtering
    \item Authentication and security system
    \item Multilingual support
\end{itemize}

These features are designed to improve the overall experience for both the employees and the users of the social security services, making the services more accessible and user-friendly.

To achieve these goals, the project utilizes two prominent web technologies: Laravel, a PHP-based web application framework, and Vue.js, a progressive JavaScript framework for building user interfaces.

The rest of the document is organized as follows: Chapter 2 provides a review of the relevant literature, while Chapter 3 outlines the project overview, objectives, and scope. Chapters 4 and 5 delve into the technologies used and system design, respectively. Implementation details are discussed in Chapter 6, followed by testing and evaluation in Chapter 7. Finally, the conclusion is presented in Chapter 8, and references and appendices can be found in Chapters 9 and 10.
