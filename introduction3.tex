\chapter{Introduction}
\section{General Introduction}

The rapid advancements in information technology have transformed the way organizations operate and manage their processes. One such domain that has witnessed significant growth is the development of web applications for various purposes. In particular, web applications have become essential tools for improving the efficiency of service delivery and reducing operational costs for many organizations, including government institutions.

\section{Problematic and Objectives}
The National Social Security Fund ( CNAS : Caisse Nationale des Assurances Sociales des Travailleurs Salariés) in Algeria is one such organization that can benefit from the adoption of web-based solutions as it is responsible for providing a range of social security services to Algerian citizens, including health insurance, retirement benefits, and unemployment benefits. The organization serves a large number of people and has several applications that we will talk about in upcoming chapters. 

Although the current system of managing queues at CNAS helps with the organization and the process of the work, it has proved to be inefficient and time-consuming for both the employees and the beneficiaries of social security, and it has been struggling to keep up with the increasing demand. For instance, imagine coming all the way to CNAS and having to wait for an hour just to get information about a document, knowing that it could be obtained in seconds through a web-based solution. This highlights the inefficiency of the current system, which is not only time-consuming but also inconvenient for the beneficiaries who have to take time off from work to visit CNAS. A web-based solution that streamlines the appointment management process will save time and effort for both the employees and the beneficiaries and will enhance the overall efficiency of the services provided by CNAS.

Therefore, the objective of this project is to create a web application that streamlines the appointment management process to improve the overall efficiency of the services provided by CNAS. The proposed web application has a key feature that enables users to choose the service and the task they want to do at CNAS before booking an appointment. At the beginning of the user's journey through the application, they are prompted to complete a questionnaire that helps generate a personalized Checklist List of the necessary documents and steps they need to complete in order to achieve their goal. This questionnaire feature streamlines the process for the user by providing clear guidance and ensuring that no important documents or steps are missed. This feature ensures that the user is directed to the appropriate service desk for their needs, reducing the time wasted on unnecessary visits and allowing users to access all the necessary information online and plan their appointments accordingly.

Moreover, this web application will include a range of features designed to enhance the appointment management process, including the ability to track the status of appointments and documents, a reminder and notification system, customization of appointments and schedules, an authentication and security system, and multilingual support.

%In addition, the COVID-19 pandemic has highlighted the need for remote access to services to reduce physical interactions and ensure social distancing.

\section{Dissertation Plan}

The rest of the dissertation is organized as follows: Chapter 2 provides a review of the relevant literature, while Chapter 3 outlines the project overview, objectives, and scope. Chapters 4 and 5 delve into the technologies used and system design, respectively. Implementation details are discussed in Chapter 6, followed by testing and evaluation in Chapter 7. Finally, the conclusion is presented in Chapter 8, and references and appendices can be found in Chapters 9 and 10.

