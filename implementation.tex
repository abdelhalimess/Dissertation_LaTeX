\chapter{Implementation}
\section{Chapter Overview}
In this chapter we will dive into the implementation details of the CNAS's virtual counter, and the different technologies used to build the system such as Version control tool GitHub, the web Development framework Laravel and the other web technologies that gave us the possibility to create an efficiant web based application.
\section{Introduction to Git }
Software Development requires the management of a big sets of files and assets which are in a frequent change, as developers we need a tool that allows the administration of those files and keep a constent updates on them, thats why Git came handy for giving us the possibility to do such task with ease and flexibility. At its core, Git is a tool that allows multiple people to collaborate on the same project, while keeping track of all changes made to the code and files over time.
\subsection{GitHub implementation}
In order to facilitates the Development phase for the virtual counter and to ensure efficiant version control for the code and system files, we have implemented Github in our project, which provided us with centralized repository that we can use for managing updates as well as set of collaborative tools and features. 

\medskip GitHub is a web-based platform that serves as a central repository for Git-based version control systems. It provides a range of tools and features that allow developers to collaborate on projects, manage code and track changes, and share their work with others with ease. The use of such tool facilitates the colaborative work by partitioning the development between us. We focused on the development in both its sides front-end and back-end so that we can share the work equally, and we managed to overcome the problem of sharing the code in traditional ways. 

 We made sure that every step in this particular phase is controled and followed in the project repository, we've started by creating the whole steps for making a reliable web application as well as noting the notions that we are going to need along the development. 
\subsection{Advanteges of using Github in Development}
Using a version control tool such as GitHub to manage the virual counter Development gave us so many tools and features that we used in our favor to ease the development process.

\medskip GitHub made the collaboration more easier for us, it provided us with a platform that allowed us to share code, simply by using issues and pull requests. In addition to this, code review feature enabled us to review the changes before one of us merge the code to the main source code,and this helped us to ensure that the code is high-quality code, maintainable, and bug-free.

In general, GitHub's code review tools make it simple for us to work together productively and identify problems early in the development cycle. By using these tools we could ensure high-quality and bug-free code, which will result a high-quality and a robust apllication.
\subsection{Github commands and screenshots  }
//
\subsection{Discussion on the use of Github}
In the development of our project, we have implemented Github as a version control system and a colaborative platform. Github was a crucial in this phase, it simplified the collaboration as well as partitioning the work between us, its efficiency was in its set of tools and features which helped us to focus more on the application it self rather then code problems and bugs. 

Overall, the use of GitHub was essential to the success of our project. By providing a centralized repository, collaborative features, efficient version control, and powerful code review tools, GitHub helped us work more effectively as a team and deliver a better end product.
\section {Laravel implementation}
//
\subsection {Overview of the Laravel framework}
//
\subsection {Explanation of the different components of the system implemented using Laravel}
//
\subsection {Code snippets and screenshots to illustrate the implementation details}
//
\subsection {Discussion of the challenges faced and how they were overcome}

//

\section {VueJs implementation}
//
\subsection {Overview of the VueJs framework}
//
\subsection {Explanation of the different components of the system implemented using VueJs}
//
\subsection {Code snippets and screenshots to illustrate the implementation details}
//
\subsection {Discussion of the challenges faced and how they were overcome}
//


\section {Integration of Laravel and VueJs}
//
\subsection {Explanation of how Laravel and VueJs were integrated to create the final system}
//
\subsection {Code snippets and screenshots to illustrate the integration details}
//
\subsection {Discussion of the challenges faced and how they were overcome}

//
    \section {Conclusion}
//
    \subsection {Summary of the key points covered}
//
    \subsection {Reflection on the overall implementation process}
//
    \subsection {Discussion of future work and potential improvements}

    //